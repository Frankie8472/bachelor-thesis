% !TEX root = ../thesis.tex

% set counter to n-1:
\setcounter{chapter}{0}
UNDER CONSTRUCTION

\chapter{Introduction}
\label{chap:introduction}

\section{Introduction}
\section{Problem Statement}


the background on and motivation for your research
- technological trends of the area
- open problems
- recent promising developments
- introduce more specific terminology which is not widely known.
- Provide good motivation for your work,
- explaining its technological, research or economic importance
- The motivation simply two or three good reasons are enough to
make your research important.

a summary and outline of your paper, telling readers what they should expect to find in it.

- a problem description, which is slightly more detailed than in the abstract.
- a description of your solution
- some arguments on its impacts
- key concepts and categorize its approach.

Close your introduction with a description of your pa-
per outline
- what sections it contains
- what the reader will find in each.

A proper flow
1. context
2. present your proposal
3. provide the verification
4. conclusions.

As the importance of computer science is rising, certain games that support the
developement of abstract thinking, analytical skills and decision making, are
becoming more and more interesting at an early age. Through identification and
describing relationships between items, children develop a foundation to early
math skills and basic concepts of computer science (e.g. combinatorics of finite
affine and projective spaces, the theory of error-correcting codes, etc.)
%\cite{SET}
. The pupils individual learning speed and lack of concentration, if
not receiving the right amount of attention, is another challenge by itself.
Without individual fostering, children are at high risk of losing interest.
Through this thesis the teachers should receive a tool for their pupils. Focused
on classification of objects with certain properties, the pupils get introduced
to a computer-based learning environment. There they can individually train and
improve themselves in this field. In the mean time the teachers can concentrate
on the majority of their pupils and have the opportunity to work with pupils on
an individual basis, without feeling the pressure of having to support everyone
at once.

\section*{Background}
The key contribution of computer science to general school education is rooted
in the concept of \textit{algorithmic thinking} %\cite{HKKS17}
. One way of
introducing kindergarten and primary school pupils to algorithmic thinking and
it's concepts consists in making them solve problems with and without computers.
This can be achieved using age- and knowledge-appropriate learning materials.
Several papers with corresponding online learning environments have been
proposed.
%\cite{STBLUM, SKAMP, TANGK, JWEBER}.
This work is going to be added to
the implementation of "INFORMATIK BIBER in KG und 1/2" by Jil Weber
%\cite{JWEBER}
. The focus of this work will not solely lie on the translation of
the existing teaching material to a computer-based learning environment, but
also on introducing learning methods in a gamified environment which not only
complements the teachers with their teachings but also assisting them.

\section*{Goals of the Thesis}
The main objective of this thesis consists of planning, analyzing, implementing
and testing a computer-based learning environment on the topic of
classification. The student studies the already existing implementations of
"INFORMATIK BIBER in KG und 1/2", analyses the capabilities of kindergarten kids
and first graders, develops an interactive classification tool, implements it
then on a platform compatible with the implementation of "INFORMATIK BIBER in KG
und 1/2" and conducts an evaluation with test subjects. We expect the student to
find a suitable implementation that integrates neatly into our existing system
mentioned before. The outcome of this thesis is a well-documented, stable and
reliable prototype, providing the functional elements to be used in schools.

\section{Section}

SECTION 0
