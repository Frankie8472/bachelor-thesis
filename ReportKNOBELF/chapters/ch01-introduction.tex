% !TEX root = ../thesis.tex

% set counter to n-1:
\setcounter{chapter}{0}

\chapter{Introduction}\label{ch:introduction}
\section{Introduction}\label{sec:introduction}
The importance of computer science is rising.
The development of abstract thinking, analytical skills and decision-making,
are becoming more and more interesting at an early age.
Through identification and describing relationships between items,
children develop a foundation to early math skills and basic concepts of computer science
(e.g.\ combinatorics of finite affine and projective spaces, the theory of error-correcting codes, hashing, etc.)\cite{cardgameset}.
The pupils individual learning speed and lack of concentration,
if not receiving the right amount of attention, is another challenge by itself.
Without individual fostering, children are at high risk of losing interest.
Through this thesis the teachers will be introduced to a tool for their pupils.
Focused on classification of objects with certain properties, the pupils get introduced to a computer-based
learning environment or so called game based learning.
There they can individually train and improve themselves in this field.
In the meantime the teachers can concentrate on the majority of their pupils and have the opportunity to work with
pupils on an individual basis, without feeling the pressure of having to support everyone at once.

\section{Goals of the Thesis}\label{sec:goals-of-the-thesis}
The main objective of this thesis consists of planning, analyzing, implementing
and testing a computer-based learning environment on the topic of
classification. The student studies the already existing implementations of
"INFORMATIK BIBER in KG und 1/2", analyses the capabilities of kindergarten kids
and first graders, develops an interactive classification tool, implements it
then on a platform compatible with the implementation of "INFORMATIK BIBER in KG
und 1/2"\cite{ibkg12} and conducts an evaluation with test subjects.
The implementation is going to be integrated into the existing system of "Einfach Informatik"\cite{einfachinformatik}.
The outcome of this thesis is a well-documented, stable and
reliable prototype, providing the functional elements to be used in schools.

\section{Problem Statement}\label{sec:problem-statement}
Concerning computer science in schools, new educational material is in production and some is already distributed.
With the printed teaching materials, the need for digitalized materials and exercises will come.
In related work [\ref{ch:relatedwork}] the already published digitalized exercises and
in which way they were realized is addressed.
The book \textit{"INFORMATIK BIBER in KG und 1/2"}\cite{ibkg12} was published recently and only parts have been digitalized yet.
In this thesis the part where children learn how to identify and compare properties of different objects is covered.

The focus though will not solely lie on the translation of the existing teaching material to a
computer-based learning environment, but also on introducing learning methods in a gamified environment which
not only complements the teachers with their teachings but also assisting them.

The main questions we are going to ask are:

\begin{itemize}
    \item How can we reach and support our target audience?
    \item What is our target audience capable of?
    \item What kind of different digital environments are there?
\end{itemize}

\section{Background}\label{sec:background}
The key contribution of computer science to general school education is rooted in the concept of
\textit{algorithmic thinking}\cite{HKKS17}.
One way of introducing kindergarten and primary school pupils to algorithmic thinking and
it's concepts consists in making them solve problems with and without computers.
This can be achieved using age- and knowledge-appropriate learning materials.

\section{Motivation}\label{sec:motivation}
Computer science provides already an easier way than other topics for teaching in the digitalized world.
As one can at least guess the huge potential to make use of the students daily enjoyments while still teaching
them the necessary things, it would be a loss to the teaching world not to explore and evaluate this kind of teaching
method. Especially game based learning as gamified learning is already being explored and evaluated.
The aspect of integrating this method as a helper/assistant to the teacher and not only as an additional feature in
their free time, is alone very tempting to explore in my opinion.

\section{Outline}\label{sec:outlook}
In the following chapters related work in the field of game based learning will be addressed,
and the target audience as well as electronic devices will be analyzed for their requirements and capabilities.
Furthermore, the created learning environment based on game based learning is explained in detail, its evaluation and
future applications and improvements.
