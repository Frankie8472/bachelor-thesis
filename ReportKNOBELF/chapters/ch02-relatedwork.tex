% !TEX root = ../thesis.tex

% set counter to n-1:
\setcounter{chapter}{1}

\chapter{Related Work}\label{ch:relatedwork}
In this section we are going through some previous works in a the same area.
Starting with mainly publications, in a second part we take look at gamification
and game based learning and in a third part we look at existing educational software
and highlight their pros and cons to build on the experiences made there.

\section{Previous work}\label{sec:previous-work}
Several papers with corresponding online learning environments based on the teaching materials "Einfach Informatik"\
\cite{ei56, ei79dat, ei79strat} have been proposed by ETH Students\cite{stblum, skamp, tangk, jweber}.

The learning environments they created/extended are mainly based on adding gamification elements to digitalized tasks
instead of creating a game based environment. This works to some extent as the texts and exercises can be used from the book.

Their environment ist based on TypeScript and accessible through the homepage \textbf{Einfach Informatik}\cite{einfachinformatik}.

\section{Gamification and Game Based Learning (GBL)}\label{sec:gamification-and-game-based-learninggbl}
\subsection{Gamification}\label{subsec:gamification}
Gamification reflects a social phenomenon arising with a generation of digitally literate population.
It has been defined as the use of "game-based mechanics, aesthetics, and game thinking to engage people,
motivate action, promote learning, and solve problems"\cite{kapp2013gamification}.
This includes digital game mechanics, but is not limited to, avatars, badges, points, levels,
leaderboard, virtual rewards, and storyline or even quests.
There is also an aspect to game elements that allow for social interaction between players and the aquisition of
critical thinking skills, which are essential in learning.

Digital video game elements that are used in the pedagogical context promote task engagement, increase motivation,
and enforce desirable learning behavious.
The rationale behind deploying video game elments in an educational context is that they have already captured the
attention of millions of people all over the world.

\subsection{Game-Based-Learning}\label{subsec:game-based-learning}
Game based learning describes an approach to teaching,
where students explore relevant aspects of games in a learning context.
Generally, it is designed to balance subject matter with gameplay and the ability of the player
to retain and apply said subject matter to the real world.

Good game-based learning applications can draw you into virtual environments that look and feel familiar and relevant.
Within an effective game-based learning environment, you work towards a goal,
choosing actions and experiencing the consequences of those actions along the way.
Making mistakes in a risk-free setting, and through experimentation,
you actively learn and practice the right way to do things.
This keeps you highly engaged in practicing behaviors and thought processes that you can easily transfer from the
simulated environment to the real life\cite{gal}.

\subsection{Gamification vs. Game-Based-Learning}\label{subsec:gamification-vs.-game-based-learning}
While game based learning is similar to gamification, it is a different breed of learning experience.
Gamification takes game elements and applies them to a non-game setting.
Game-based learning is designed to balance subject matter with gameplay and the ability of the player to retain,
and apply said subject matter to the real world.

\section{Existing Educational Software}
The idea of propagating games based on game based learning is often touched but rarely integrated in schools.
Sometimes they have more incommon with normal computer games and the learning aspect is only discovered when specific looked at.
En example of such a company is the learning company\cite{tlc}.

The learning company produced a grade-based system of learning software and tools to improve productivity.
It was known for its games like Reader Rabbit\cite{readerrabbit} and OutNumbered!\cite{outnumbered}.

Reader Rabbit is an educational game franchise and a series aimed at children from infancy to the age of eight.

OutNumbered! is an educational computer game software aimed at children ages seven to fourteen and
is designed to teach children mathematical computation and problem solving skills.

Both games have a storyline with a specific designed environment to include minigames with different topics.

The games teach language arts including basic skills in reading and spelling, and mathematics. The main character in all the titles is named "Reader Rabbit".
The games teach language arts including basic skills in reading and spelling, and mathematics. The main character in all the titles is named "Reader Rabbit".

pro: player didnt recognize he was learning something
con: every level needed the same level of knowledge in every category


\section{Checklist}
X - list of research works that are related to your paper necessary to show what has hap-
pened in this field.
X - critique of the approaches in the literature necessary to establish the
contribution and importance of your paper.

X distinguish and describe all the different approaches to the problem.

Critiquing the major approaches of the background
work will enable you to identify the limitations of the
other works and show that your research picks up where
the others left off. This is a great opportunity to demon-
strate how your work is different from the rest; for ex-
ample, show whether you make different assumptions
and hypotheses, or whether your approach to solving the
problem differs.

DIE FRAGE: ein neues framework, einfach zu lernen, ja nein? aufwand, vorteile, nachteile

irgendwo mus ine, dass au lehrer abgneigt sind gege elektronik im taegliche gebruch. jedoch isches fact, 
dass dmehrheit vode chinder scho im junge alter mit dere war konfrontiert wird. de taeglichi gebruch isch sowieso da. 
wieso noed gad als guets bispil vorusgah und ihne spil schmackhaft mache wo ihne spass mached und si demit passiv lerned. 
motivation isch hoecher
chind choennd scho hoechduetsch, bevor si ueberhaupt schwizerduetsch richtig choennd, will si mit duetscher sprach im gebruch vo elektronik
konfrontiert werded. warum noed gad bruche als au soettigs und dete de vorteil drus zieh?! mr chan als bispil voragah ide informatik
da das ersch jetzt gross ufchunnt, bi etablierte lehrmittel wird das schwiriger vorallem wills denne oftmals heisst, 
mit dem hanich scho glernt und das het funktioniert
