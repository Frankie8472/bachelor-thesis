% !TEX root = ../thesis.tex

% set counter to n-1:
\setcounter{chapter}{1}

\chapter{Related Work}
\label{chap:relatedwork}
UNDER CONSTRUCTION

In this section we are going through some previous works in a the same area.
Starting with mainly publications, in a second part we take look at gamification
and game based learning and in a third part we look at existing educational software
and highlight their pros and cons to build on the experiences made there.

\section{Previous work}
Sonja Tabea Blum
Kevin Tang
Sarah Kamp
Jil Weber

\section{Gamification and Game Based Learning (GBL)}
4
5
6
7
8

\section{Existing Educational Software}
%https://einfachinformatik.inf.ethz.ch/kindergarten
%https://en.wikipedia.org/wiki/The_Learning_Company
%https://en.wikipedia.org/wiki/Reader_Rabbit
%https://en.wikipedia.org/wiki/OutNumbered!

pro: player didnt recognize he was learning something
con: every level needed the same level of knowledge in every category


\section{Checklist}
X - list of research works that are related to your paper necessary to show what has hap-
pened in this field.
X - critique of the approaches in the literature necessary to establish the
contribution and importance of your paper.

X distinguish and describe all the different approaches to the problem.

Critiquing the major approaches of the background
work will enable you to identify the limitations of the
other works and show that your research picks up where
the others left off. This is a great opportunity to demon-
strate how your work is different from the rest; for ex-
ample, show whether you make different assumptions
and hypotheses, or whether your approach to solving the
problem differs.

DIE FRAGE: ein neues framework, einfach zu lernen, ja nein? aufwand, vorteile, nachteile

irgendwo mus ine, dass au lehrer abgneigt sind gege elektronik im taegliche gebruch. jedoch isches fact, 
dass dmehrheit vode chinder scho im junge alter mit dere war konfrontiert wird. de taeglichi gebruch isch sowieso da. 
wieso noed gad als guets bispil vorusgah und ihne spil schmackhaft mache wo ihne spass mached und si demit passiv lerned. 
motivation isch hoecher
chind choennd scho hoechduetsch, bevor si ueberhaupt schwizerduetsch richtig choennd, will si mit duetscher sprach im gebruch vo elektronik
konfrontiert werded. warum noed gad bruche als au soettigs und dete de vorteil drus zieh?! mr chan als bispil voragah ide informatik
da das ersch jetzt gross ufchunnt, bi etablierte lehrmittel wird das schwiriger vorallem wills denne oftmals heisst, 
mit dem hanich scho glernt und das het funktioniert
