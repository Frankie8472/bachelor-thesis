% !TEX root = ../thesis.tex

% set counter to n-1:
\setcounter{chapter}{1}

\chapter{Related Work}\label{ch:relatedwork}
In this section we are going through some previous works in the same area.
Starting with mainly publications, in a second part we take look at gamification
and game based learning and in a third part we look at existing educational software
and highlight their pros and cons to build on the experiences already made there.

\section{Previous work}\label{sec:previous-work}
Several papers with corresponding online learning environments based on the teaching materials "Einfach Informatik"\
\cite{ei56, ei79dat, ei79strat} have been proposed by ETH Students\ \cite{stblum, skamp, tangk, jweber}.

The learning environments they created/extended are mainly based on adding gamification elements to digitalized tasks
instead of creating a game based environment.
This works to some extent as the texts and exercises can be used from the book.

Their environment is based on TypeScript, the Angular framework and accessible through the homepage \textbf{Einfach Informatik}\cite{einfachinformatik}.

It contains a lot of information for both students and teachers.
Information about the task is mainly obtained via text.
Thus one requirement is being able to read or being accompanied by someone who is able to.

Often teacher can upload exercises or create tasks.
This can be seen as a disadvantage as then the teacher has to focus on doing something other than care for the majority
of his class.

Exercises are sorted by textbook chapter, which is in my opinion a valid strategy to use the learning environment as an
addition to the textbook.
Each topic contains exercises, as well as exams, where students can solve
them and sum up the skills they were supposed to acquire in this chapter.
The division of the exercises into different difficulty levels offers each student level adjusted exercises by knowing
in advance which exercises require a basic level of understanding and which ones are for the more advanced students.
The main goal tries to focus on the engagement of the students interest on the subject which is good.

\subsection{Angular}\label{subsec:angular}
Angular is an open-source web application framework based on TypeScript.
It is maintained by Google and offers lots of built-in functionalities.
It uses HTML and CSS for the view of the application and TypeScript for the model and the controller.
As Google is supervising the framework, it is continuously developed and tested,
which makes it a stable Framework giving lots of hands for fast development.

\section{Gamification and Game Based Learning (GBL)}\label{sec:gamification-and-game-based-learninggbl}
\subsection{Gamification}\label{subsec:gamification}
Gamification reflects a social phenomenon creating a generation of digitally literate population.
It has been defined as the use of "game-based mechanics, aesthetics, and game thinking to engage people,
motivate action, promote learning, and solve problems"\cite{kapp2013gamification}.
This includes digital game mechanics. It is not limited to, avatars, badges, points, levels,
leader board, virtual rewards, and story line or even quests.
There is also an aspect to game elements that allow for social interaction between players and the acquisition of
critical thinking skills, which are essential in learning.

Digital video game elements that are used in the pedagogical context promote task engagement, increase motivation,
and enforce desirable learning behaviour.
The rationale behind deploying video game elements in an educational context is that they have already captured the
attention of millions of people all over the world.

\subsection{Game-Based-Learning}\label{subsec:game-based-learning}
Game based learning describes an approach to teaching,
where students explore relevant aspects of games in a learning context.
Generally, it is designed to balance subject matter with game play and the ability of the player
to retain and apply said subject matter to the real world.

Good game-based learning applications can draw the user into familiar virtual environments..
Within an effective game-based learning environment, it is common to work towards a goal,
choosing actions and experiencing the consequences of those actions along the way.
Making mistakes in a risk-free setting, and through experimentation,
it is actively learned and practiced the right way to do things.
This keeps the user highly engaged in practicing behaviors and thought processes that he can easily transfer from the
simulated environment to the real life\cite{gal}.

\subsection{Gamification vs. Game-Based-Learning}\label{subsec:gamification-vs.-game-based-learning}
While game based learning is similar to gamification, it is a different breed of learning experience.
Gamification takes game elements and applies them to a non-game setting.
Game based learning is designed to balance subject matter with game play and the ability of the player to retain,
and apply said subject matter to the real world.

\section{Existing Educational Software}\label{sec:existing-educational-software}
The idea of propagating games based on game based learning is often touched but rarely integrated in schools.
Sometimes they have more in common with normal computer games and the learning aspect is only discovered when specificly looked at.
A good example of a such company is the learning company\cite{tlc}.

The learning company produced a grade-based system of learning software and tools to improve productivity.
It was known for its games like Reader Rabbit\cite{readerrabbit} and OutNumbered!\cite{outnumbered}.

Reader Rabbit is an educational game franchise and a series for children from infancy to the age of eight.

OutNumbered! is an educational computer game software aimed at children from the age seven to fourteen and
is designed to teach children mathematical computation and problem solving skills.

Both games have a story line with a specific designed environment which includes mini games and challenges covering
various topics not only focusing on one subject.

The games teach language arts including basic skills in reading and spelling, and mathematics.
They are designed to be played alone. Not within the classroom but as additional homework or
recreational activities over the holidays to not forget the freshly learned subjects.

It had great success depending on the perspective. The kids were sitting at the computer over the holidays in summer
and "learning" instead of playing outside. Kids see such games as what the games are trying to be, games without noticing
the learning aspect. Though, it is difficult to create one game for a broad spectrum of users, as the level of the topics
vary from country to county or even town to town. This becomes even more complicated if multiple topics from different
school subjects are included in one game. If a part of a game is too easy or too hard for the user you are risking
loosing his attention fast, depending on his the age.

In my opinion the rise and fall of those games was on one hand the integrated learning process in a story and mini games which
were very satisfying to the normal user, as he did not notice that he was learning.
Unfortunately, the games were not supporting the teacher in class and only supplementary.
Why should I play a game related to school when there are multiple other games with better game play,
a more interesting story line and better visual effects?
Yes, it is nearly impossible to compete with the game industry if you are not , but if you cannot compete, change the play field.
The play field normal games will never touch is the school, as they normally have no correlation to school topics.
