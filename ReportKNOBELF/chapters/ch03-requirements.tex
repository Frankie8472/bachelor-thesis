% !TEX root = ../thesis.tex

\chapter{Requirements}\label{ch:requirements}
In this section we will capture the requirements for our learning environment.
We will look into the cognitive and motor abilities of the target audience and
some elements of gamification which will be analyzed and assessed.
The existing environment, our work should be implemented in, has conditions as well.
Those will be taken into consideration while evaluating a suitable framework.

\section{Programming Environment}\label{sec:programming-environment}
\subsection{Boundary Conditions}\label{subsec:boundary-conditions}
The following conditions were extracted from the assignment:

\begin{itemize}
    \item The game should be maintainable as it is not excluded that others are going to work on it.
    \item Modularity is an important aspect as one cannot assume the final design at this point in time.
    \item It should be possible to integrate it in a Typescript/JavaScript/HTML environment (e.g.\ as a canvas) as
    all similar work up until now is mostly written in Typescript in the Angular framework.
    \item The ability to run on computers and tablets is important as a lot of schools have at least one.
    \item The game should be self-explainable as the teacher has to keep his focus on the majority of his students.
    \item Easy to learn. Support is an important aspect as it is easier to start with something new.
    A huge community correlates in most cases with good support on free open source software.
\end{itemize}

\subsection{Evaluation of Possible Solutions}\label{subsec:evaluation-of-possible-solutions}
\subsubsection{Babylon.js}
Babylon.js is a complete JavaScript framework for building 3D games.
Using WebGL for graphics, the feature set for Babylon is somewhat extensive.
Its community is said to be fairly active.
The most significant features worth mentioning are:

\begin{itemize}
    \item Support and exporting tools
    \item Game engine staples such as scene picking, collision handling, and scene graphs
    \item Particle and animation systems
    \item Performance optimizations such as frustum clipping, hardware scaling, and occlusion queries
    \item Shader, rendering, and texture systems
    \item Expansive mesh support
\end{itemize}

Babylon does not require to be installed as an internal entity on your computer.
Thus, all development can happen within the browser/code editor itself\cite{phaserad}.

\subsubsection{Pixi.js}
Pixi.js is a 2D game rendering engine intended for HTML5 games.
There are benefits of its integrated hardware acceleration.
Pixi’s focus lies not on WebGL, yet utilizes rich game content, interactive displays,
and apps that are supported on all platforms equally.
It is said that it is the way that Pixi has been built that enables for it to be a smooth, rapid, and evenly interactive rendering engine\cite{phaserad}.

\subsubsection{Melon.js}
Melon.js has a sprite-built JavaScript engine for 2D game development,
is an independent project which does not require additional libraries to work,
supports mobile type devices as well as all leading browsers,
optimization for mobile devices for motion and hardware,
in-built HTML5 audio support,
a practical physics engine to reduce the CPU usage,
a great deal of effects that would be required for creating a
functional on-line game in the browser.
Community forums is hosted on Google Groups, where you can quickly yield answers to your questions in regards to how Melon.js works or in the case of you experiencing bugs.
The documentation features several dozens of demo applications built with Melon,
some of which are open-source and can be used to learn different aspects of game development from\cite{phaserad}.

\subsubsection{Phaser}
Phaser is a free and open source JavaScript/TypeScript framework which puts a focus on letting coders make games quickly.
The system works primarily with Canvas and WebGL, letting programmers easily build substantial games for both
desktop and mobile browsers.

It is said that phaser has an active community that regularly participates on the forum, Slack, and Discord channels.

One of the biggest benefits of this engine is that it is a fully-featured engine,
so it isn’t restricted to doing just one thing.
Here’s a list of some of the feature sets provided with Phaser:

\begin{itemize}
    \item Built on WebGL and canvas
    \item Preloader system
    \item Physics features
    \item Sprite and animation handling
    \item Particle system
    \item Camera, input, and sound systems
    \item Tilemap support
    \item Device scaling support
    \item Plugin availability
\end{itemize}

Phaser’s preloader makes it easy for developers to load their game assets and have them automatically handled.
That way, one des not have to waste time writing extensive code for each part of the game\cite{phaserad}.

\subsubsection{Evaluation of the Frameworks}
Considering the boundary conditions one can see that the TypeScript support, a lot of code examples,
an active community for support and multiple browser support is most important for this work.
It is trivial to see that the phaser framework fulfils these conditions to a fairly good extent.

\section{Boundary Conditions of the Problem Statement}\label{sec:boundary-conditions-of-the-problem-statement}
Considering the problem statement the user has to learn how to identify and compare properties of objects.
As these object should just be placeholders, the simplest objects can be used for that purpose (e.g.\ geometrical objects).
To make the task of comparing more difficult the objects can have more than one property.
The user should train the task at hand on different levels of difficulty.
There should be easy levels for inexperienced users and hard levels for more advanced users.
The aspect of hashing in the context of sorting these objects with limited available space must be included.

\section{Boundary Condition of the Target Audience}\label{sec:boundary-condition-of-the-target-audience}
The target audience are going to be children between the age of 5 and 8.
They cannot be expected to be able to read or write. Numbers from 1 to 5 should be possible though.
The motoric abilities of children with tablets are rather advanced in contrast to using a computer with a mouse.
So a touch screen of any kind is not a problem but the handling with a keyboard of a mouse cannot be assumed.

\section{Additional Conditions: Elements of Gamification}\label{sec:additional-conditions:-elements-of-gamification}
To enhance the user experience and to influence his behaviour the following elements of gamification should be implemented:
\begin{itemize}
    \item Different levels for different experienced users.
    \item A reward system for instant gratification.
    \item Global progress tracking to keep track of your success.
    \item The gratification should be visualized and animated as this gives the user more satisfaction
    \item Tracking of correct and incorrect choices made to have a live tracking, which adds motivation and
    stress on the same time (higher difficulty).
    \item Time limits to put the user under stress.
    \item Sound can add a valuable replay factor.
    \item If possible there should be an aspect of a story/timeline. This has the potential to captivate the user on a
    whole different level.
    \item A level menu or a map is useful addition or replacement for a story.
    \item A full screen option helps to stay focused and not get distracted by other visuals from your environment
    \item A pause menu is important for the user as he can always be distracted.
    Without a pause menu the user may get frustrated and fed up with the game.
\end{itemize}
