% !TEX root = ../thesis.tex

% set counter to n-1:
%\setcounter{chapter}{1}

\chapter{Design}
\label{chap:design}

In this section we will capture the requirements for our learning environment.
We will look into the cognitive and motor abilities of the target audience and
some elements of gamification which will be analyzed and assessed.
The existing environment our work should be implemented has conditions as well.
Those will be taken into consideration while evaluating a suitable framework.

\section{Requirements}
\subsection*{Target Audience}
\subsection*{Elements of Gamification}
\subsection*{Additional Conditions}
\subsection*{Framework}

human
- kinder koennen nicht lesen und schreiben, keine groesseren zahlen und keine buchstaben
-

Game
- capturing name
- objects categories nad subcategories
- easy level
- easy level under harder difficulties
- limited sorting -> hashing
- set easy
- set hard
- overview
- gamification
    - track progress
        - time based
        - correct & wrong
        - visual gratification
    - have fun dont recognize you are learning
        - optics
            - randomization and more options for variety
        - sound
        - story like
    - menu
    - levelmenu
    - fullscreen
    - flow
        - exitbutton
        - returnbutton
    -


\section{Checklist}
- explicitly describe all the hypotheses and assumptions of the environment on which the problem will be stated.
- Put good effort in realizing all explicit and implicit assumptions that you make, and clearly state them.
It is important to provide support for
your assumption choices. The more valid and acceptable
your assumptions are, the more valid and acceptable your
work will be.
The system model section should always have a figure.
The figure should demonstrate the parameters of your
system model. Prepare the figure so that it can later be
reused or enhanced to demonstrate your solution.

Often, this section is merged with the system model.
State your problem clearly. Be as exact as possible into
stating what the question of the problem is. It reflects
poorly upon an author if he cannot describe or does not
know what problem his solution addresses. But most im-
portantly, it will be easier for successive researchers to
classify your work.
