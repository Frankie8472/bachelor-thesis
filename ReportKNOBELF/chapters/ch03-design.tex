% !TEX root = ../thesis.tex

% set counter to n-1:
%\setcounter{chapter}{1}

\chapter{Design}
\label{chap:design}

In the system model section, you explicitly describe all the
hypotheses and assumptions of the environment on which
the problem will be stated. Put good effort in realizing
all explicit and implicit assumptions that you make, and
clearly state them. It is important to provide support for
your assumption choices. The more valid and acceptable
your assumptions are, the more valid and acceptable your
work will be.
The system model section should always have a figure.
The figure should demonstrate the parameters of your
system model. Prepare the figure so that it can later be
reused or enhanced to demonstrate your solution.

Often, this section is merged with the system model.
State your problem clearly. Be as exact as possible into
stating what the question of the problem is. It reflects
poorly upon an author if he cannot describe or does not
know what problem his solution addresses. But most im-
portantly, it will be easier for successive researchers to
classify your work.

\section{Section}

SECTION 2
