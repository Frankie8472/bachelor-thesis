% !TEX root = ../thesis.tex

\chapter{Evaluation}\label{ch:evaluation}
In this section the environment the game was tested in will be presented and analysed.
First every game has to be tested on different devices for code errors or unintended behaviour.
Second and third a target and various audience will test the game and significant reaction and behaviour will be stated.

\section{Debugging}
\subsection{Devices}
The game was tested on the following devices and browsers:

\textbf{Browsers}
\begin{itemize}
    \item Firefox
    \item Chrome
    \item Microsoft Edge (only on devices with Windows OS)
    \item Safari (only on iOS devices)
\end{itemize}

\textbf{Devices}
\begin{itemize}
    \item Computer with Windows 10
    \item Computer with Apple iOS
    \item Tablet with Apple iOS
    \item Tablet with Android
    \item Smartphone with Android
    \item iPhone
\end{itemize}

\subsection{Differences and obstacles}
On Windows and Android devices the browser based application run without any troubles.
On Apple devices there is an issue with the fullscreen option.
As soon as the fullscreen is engaged, the touch functions of the operating system interferes with the touch input in the
browser window. This is a common problem with apple devices and the only known solution to work so far is to just
minimize the header bar.
This is an acceptable solution considering all the safety restriction apple applies to its users and developers.

\subsection{Unexpected Bugs}
With the various test audience some unexpected bugs are found. The most fascinating are mentioned here.

One would not expect the users to something that is not goal related.
For example dragging an object not to its intended destination but to the screen/window boundary and even further.
To ensure the object can still be accessed through dragging,
a restriction on dragging the object outside of the windows had to be added.

After a goal is reached and the scene transition sets in, one could expect the user to wait until a new scene is loaded.
As this is not the case further restriction on the user input had to be added.

The users thinking becomes faster the longer he plays the game and so are his interaction/inputs.
Thus different locks on animations had to be placed, so that while some animation is in progress not another animation can
be triggered.

It was fascinating to see that users have a different thickness of their finger.
For users with a thicker fingers, sometimes the object size was too small to touch and still observe the now dragged object.
Thus adjustments had to be made to the size of objects.

\section{Target Audience Test}
The created learning environment was tested on two groups:
\begin{enumerate}
    \item 6 test subjects; age between 4 and 7; advanced knowledge on handling electronic devices
    \item 5 test subjects; age between 4 and 7; less to none knowledge on handling electronic devices
\end{enumerate}

The first group had no problem at all interacting with the browser based application.
The second group needed more time to get a feeling of the game mechanics but after the adjustment time the difference of
the two groups was negligible.

In the end both test groups were eager to play and explore the newly presented learning environment.

In the first levels the perfect score was almost always reached on the first try.

The younger ones had the most fun with the middle staged levels with the falling objects but were eager and curious
to explore the harder levels. With the support of the older ones or even an adult playing the easy pairing game
was possible and understandable.

The older ones were most eager to understand and beat the hardest level and sometimes found a possible solution
faster than the adult test group.

What was fascinating that almost every test subject knew what to do in the level with restricted space but
some had different tactics to approach it. The two tactics used were:
\begin{itemize}
    \item Fixating a category and finding the solution with trial and error.
    \item Sorting the objects in the field above the boxes after a category.
\end{itemize}

Some test subjects wanted to play more even after playing all the levels and some of them even formed a group and
were discussing possible solutions, sharing their thoughts.

\section{Various Audience Test}
\begin{enumerate}
    \item 11 test subjects; age between 18 and 40; daily usage of electronic devices (computer, tablet, smartphone)
    \item 13 test subjects; age between 30 and 70; only some knowledge in how to handle a smartphone
\end{enumerate}

The first test group had no problem at all finding out how to interact with the browser based application.
All of them figured out all tasks on their own. As in the first test run the time limit on the last level was somewhat
too small, at first no one managed to beat it at all. Fascinating to see was that 5 test subjects did not want to stop
playing until they beat the game, what they did in the end.

The second test group needed some adjustment time to figure out how to interact with the browser based application.
On some one could see that the motoric abilities were not trained as well. One could assume that this were the older
test subjects. On the contrary, yes there were two test subject in the older spectrum, but three younger test subjects
had not used their smartphone in that way in a long time and some of their motoric functions deteriorated
in a fascinating way.
Furthermore almost all of them had trouble understanding the level with sorting objects in boxes with restricted space
but not with the other levels.
Though the ones who immediately understood the task on hand had problem understanding other tasks others had no problems with.

There were two teachers (one preschool and one kindergarden) who tested the game.
One of them was not found of tablets in kindergarden and opposed the idea in the beginning.
After she found out that the application was not intended to replace some topics or ways they are teaching but
to enhance their spectrum and ability to deal with their pupils on their respective levels,
she wanted to give the idea a chance. The aspect of group work was also very welcomed.
