% !TEX root = ../thesis.tex

\chapter{Evaluation}
\label{chap:evaluation}

Depending on your budget and available time, you may
have performed simulations or even some experiments. In
either case, it is important to describe the environment
of your experiments or simulations. This includes stating
the parameters and conditions of the environment (sim-
ulated or real), what measurements were taken and how
they were taken.
You need to establish the fact that your simulation or
experiment results are statistically stable, meaning that
they are representative of the space of possible results.
Performing experiments and simulations is a subtle mat-
ter, always putting the validity of your data at risk in
many aspects. Before you perform the simulation or ex-
periment, educate yourself on how to perform simula-
tions, how to interpret the results and how to present
them in graphs and figures.
Each figure (or graph) should be well explained. Dedi-
cate at least one paragraph for each figure. Describe what
the reader sees in each figure and what he should notice.
Moreover, reason on the results—are they the way they
were expected to be? Avoid giving tables of numerical
data as means of presenting your results.
Compare the performance of your solution to the per-
formance of one or two competing solutions. Usually,
when you simulate or experiment on your solution, you
simulate it in contrast to a competing solution. You have
to make sure that the test scenarios are fair and make an
argument about the fairness of your comparison in the
paper.
A special case of experiment is the usability test. Many
times, although the performance of a solution can be im-
pressive, the applicability of it can be minimal. Usability
test is a type of experimentation, which determines the
acceptance of a solution by end-users or its suitability for
certain applications. Usually, papers on software product
solutions contain usability tests.

\section{Section}

Section 4
