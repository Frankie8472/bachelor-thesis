% !TEX root = ../thesis.tex

\chapter*{Abstract}

As the importance of computer science is rising, certain games that support the
developement of abstract thinking, analytical skills and decision making, are
becoming more and more interesting at an early age. Through identification and
describing relationships between items, children develop a foundation to early
math skills and basic concepts of computer science (e.g. combinatorics of finite
affine and projective spaces, the theory of error-correcting codes, hashing, etc.)
\cite{SET}. The pupils individual learning speed and lack of concentration, if
not receiving the right amount of attention, is another challenge by itself.
Without individual fostering, children are at high risk of losing interest.
Through this thesis the teachers will be introduced to a tool for their pupils.
Focused on classification of objects with certain properties, the pupils get
introduced to a computer-based learning environment. There they can individually
train and improve themselves in this field. In the mean time the teachers can
concentrate on the majority of their pupils and have the opportunity to work with
pupils on an individual basis, without feeling the pressure of having to support
everyone at once.
It has shown, that gamification has enormous potential. It takes time to learn how
to create games. Once this hurdle is overcome, creating further games is not as
hard as one may think. On the created test surface the test subjects have shown
more concentration, individual work behaviour and willingness to learn than in
the whole group.
Keywords : phaser 3, typescript, debugging, testing, from a topic to the game,  how to tackle a new framework
