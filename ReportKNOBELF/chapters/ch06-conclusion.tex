% !TEX root = ../thesis.tex

\chapter{Conclusion}
\label{chap:conclusion}

limitation
- design
- knowledge of the framework and examples for complete funcitonality "der gedanke dahinter"
-

disadvantages
- lehrer mues iharbeite will beschrenkt durch keine zahlen und woerter
- durch eine webaplikation schwer zu beschränken auf gewissen geräten (apple)


\section{checklist}
The conclusions section, similar to the introduction and
related work sections, serves two purposes. The first is
to elaborate on the impacts of using your approach. The
second is to state limitations or disadvantages of your so-
lution, thus enabling you to provide directions for future
research in the field.
