% !TEX root = ../thesis.tex

\chapter{Conclusion}\label{ch:conclusion}
The idea of not replacing but enhancing a teachers ability to teach pupils on their respective levels with
game based learning has potential. It is an option worth evaluating in detail as the interest and curiosity of
the test subjects (age independently) on both sides was in my opinion by far not negligible.
The adapting behaviour concerning task recognition and group building has shown that electronic
devices are not bad tools, only when used the wrong way.

Interesting to see is that an application intended for a very young age is able to foster and fascinate even adults.

\section{Limitation}
One big hurdle will be the time investment to create an almost ideal prototype to convince teachers that using this learning
method is not a downgrade from the current way of learning in school/kindergarten.

Through my thesis I realized that a lot can be achieved by having the right background/story and design.
For that the modularity comes into play.
With the cornerstones laid the right environment can be created and hopefully integrated in my work.

As much as I would like to say I understood all the things I did, even in the end, I discovered new techniques and features of
the phaser framework. Those new insights could further tighten, defragment and increase the readability and maintainability of my code.
But to which extent I dare not to make a guess.

Further TypeScript support will be added with the upcoming fourth version of phaser.

\section{Future Improvements}
Furter improvements are clearly the integration of the knowledge gained towards the end of my work as well as coming up
with a specific design and environment for the topic the game is used in combination with the other learning material.

