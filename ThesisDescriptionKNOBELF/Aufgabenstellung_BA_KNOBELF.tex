\documentclass[a4paper,oneside,11pt,final]{scrartcl}
\usepackage[T1]{fontenc}
\usepackage[latin1]{inputenc}
\usepackage[english]{babel,varioref}

\usepackage[newcommands]{ragged2e}
\usepackage{lmodern}
\usepackage{textcomp,latexsym,url}
\usepackage{array,booktabs}
\usepackage[final]{microtype}
\usepackage{graphicx}
%\usepackage{mathtools}
\usepackage{enumerate}
%\usepackage{amsmath} \usepackage{amsfonts} \usepackage{amssymb}
\usepackage{lastpage}
\usepackage{csquotes}
\usepackage{xcolor}

\usepackage[% colorlinks,% linkcolor=blue!75!black,% citecolor=blue!75!black,%
 pdfauthor={Chair of Information Technology and Education at ETH Zurich},%
 pdftitle={Bachelor Thesis}]{hyperref}
\usepackage[nameinlink]{cleveref}


\usepackage[left=3cm,top=3cm,right=3cm]{geometry} 

\newcommand{\ethz}{ETH Zurich}
\newcommand{\ABZ}{Center for Informatics Education of ETH Zurich (ABZ)}
\newcommand{\ITE}{Chair of Information Technology and Education}
\newcommand{\student}{Franz Knobel}
\newcommand{\professor}{Prof. Dr. Juraj Hromkovi\v{c}}
\newcommand{\cosupervisor}{Dr. Elizabeta Cavar}
\newcommand{\cosupervisorfunction}{TBD}
\newcommand{\xlogo}{\emph{XLogoOnline}}

\pagestyle{empty}

%Header, Footer:
\usepackage{fancyhdr}
\pagestyle{fancy}
\lhead{}
\chead{}
\rhead{}
\lfoot{\ITE}
\cfoot{}
\rfoot{Page \thepage/\pageref{LastPage}}
\renewcommand{\headrulewidth}{0.0pt}
\renewcommand{\footrulewidth}{0.4pt}

\hyphenation{PrimaLogo}

\begin{document}

\noindent\includegraphics[height=20mm]{ETHlogo_13} %\hfill \includegraphics[height=15mm]{Deplogo_13} 

%\noindent\includegraphics[height=30mm]{eth_logo_kurz_pos} %\hfill
%\includegraphics[height=15mm]{Deplogo_13} 

%\begin{tabular}{p{0.6\textwidth} p{0.4\textwidth}} \vspace{0pt}
% \includegraphics[width=0.4\textwidth]{eth_logo_kurz_pos} & \vspace{0pt}
% \includegraphics[width=0.3\textwidth]{Deplogo_13} \end{tabular}

\vspace{2cm}
\begin{center}
	{\huge\bfseries Bachelor Thesis} \\
	Course Unit: 252-0500-00\\
	10 ECTS credits\\
	\huge Project Description
	\vspace{3mm}
\end{center}
\vspace{1cm}

\section*{Topic} 
Gotscha! Meet an interactive learning environment with focus on object
identification and classification. It supports the teacher by challenging pupils
on their respective level. Concentrate on the mass whilst foster on an
individual level! 

\section*{Student}
\student{} (10-932-879), enrolled in the Computer Science Bachelor Programme at
\ethz.

\section*{Supervision \& Grading}
Responsible Lecturer: \professor \\
Supervisor: \cosupervisor

\section*{Time-Frame}
Assignment date: Wednesday, July 31, 2019\\
Submission due: Friday, January 31, 2020

\newpage

\section*{Motivation}
As the importance of computer science is rising, certain games that support the
developement of abstract thinking, analytical skills and decision making, are
becoming more and more interesting at an early age. Through identification and
describing relationships between items, children develop a foundation to early
math skills and basic concepts of computer science (e.g. combinatorics of finite
affine and projective spaces, the theory of error-correcting codes, etc.)
\cite{SET}. The pupils individual learning speed and lack of concentration, if
not receiving the right amount of attention, is another challenge by itself.
Without individual fostering, children are at high risk of losing interest.
Through this thesis the teachers should receive a tool for their pupils. Focused
on classification of objects with certain properties, the pupils get introduced
to a computer-based learning environment. There they can individually train and
improve themselves in this field. In the mean time the teachers can concentrate
on the majority of their pupils and have the opportunity to work with pupils on
an individual basis, without feeling the pressure of having to support everyone
at once.

\section*{Background}
The key contribution of computer science to general school education is rooted
in the concept of \textit{algorithmic thinking} \cite{HKKS17}. One way of
introducing kindergarten and primary school pupils to algorithmic thinking and
it's concepts consists in making them solve problems with and without computers.
This can be achieved using age- and knowledge-appropriate learning materials.
Several papers with corresponding online learning environments have been
proposed \cite{STBLUM, SKAMP, TANGK, JWEBER}. This work is going to be added to
the implementation of "INFORMATIK BIBER in KG und 1/2" by Jil Weber
\cite{JWEBER}. The focus of this work will not solely lie on the translation of
the existing teaching material to a computer-based learning environment, but
also on introducing learning methods in a gamified environment which not only
complements the teachers with their teachings but also assisting them. 

\section*{Goals of the Thesis}
The main objective of this thesis consists of planning, analyzing, implementing
and testing a computer-based learning environment on the topic of
classification. The student studies the already existing implementations of
"INFORMATIK BIBER in KG und 1/2", analyses the capabilities of kindergarten kids
and first graders, develops an interactive classification tool, implements it
then on a platform compatible with the implementation of "INFORMATIK BIBER in KG
und 1/2" and conducts an evaluation with test subjects. We expect the student to
find a suitable implementation that integrates neatly into our existing system
mentioned before. The outcome of this thesis is a well-documented, stable and
reliable prototype, providing the functional elements to be used in schools.

\newpage

\section*{Tasks}
As part of the Bachelor Thesis, the student is expected to perform the following
main tasks and their subtasks:

\begin{enumerate}
	\item Concept
	\begin{enumerate}
		\item Analyze the teaching material concerning classification currently
		available for kindergarten to 2nd grade. 
		\item Familiarize yourself with web technology.
		\item Study the relevant publications of the Chair of Information
		Technology and Education in didactics of computer science.
		\item Investigate related work on the topic of teaching children to
		classify and identify properties and explore different designs. 
		\item Explore how children react to existing learning-based games (Using
		a pre-test)
		\item Explore what requirements and demands arise when children are
		confronted individually or in groups with your environment (see 3a and
		3b). 
	\end{enumerate}
	
	\item Design and implementation
	\begin{enumerate}
		\item Propose a suitable environment for your computer-based learning
		platform. (e.g. AngularJS, a game-framework, python, etc.)
		\item Implement and document a maintainable interactive evironment.
		\item Add gamification elements (e.g. a reward system, different levels)
		\item Integrate your work into Jil Weber's environment "INFORMATIK BIBER
		in KG und 1/2" \cite{JWEBER}.
		\item Test your final product in class (see 3c).
	\end{enumerate}
	
	\item Evaluation
	\begin{enumerate}
		\item Conduct a preliminary study evaluating the success rate of primary
		school children and kindergarten children in interacting with your
		learning enviroment individually and in groups. 
		\item Investigate how the reward system affects the social interaction
		between pupils.
		\item Final evaluation. Find out whether your application can fulfill
		it's purpose.
		
	\end{enumerate}
	
	\item Deliverables
	\begin{enumerate}
		\item Elaborate on your work in a well-documented report.
		\item Demonstrate the application.
		\item Give a presentation of the thesis.
	\end{enumerate}
\end{enumerate}

\newpage

\section*{Deliverables}

\noindent The student is expected to manage the project, comply with rigorous
scientific standards, and to apply sound software engineering procedures in
order to successfully design, develop and test the platform.

\noindent We expect an extensive and detailed documentation of all steps of the
project. The student provides both a high-level documentation (i.e., a
description of the inner workings and communication protocol used in the
extension) as well as a low-level documentation in the form of suitable comments
in the source code.

\section*{Grading Criteria}
The Bachelor Thesis will be graded according to the \textit{Guidelines
Bachelor's Thesis} \cite{IR}. Major emphasis will be placed on the quality of
\textit{objectives and scope}, \textit{scientific approach}, \textit{design},
\textit{implementation}, \textit{reflection}, \textit{autonomy},
\textit{learning aptitude}, \textit{creativity}, \textit{documentation},
\textit{presentation}, and \textit{time management}.

\begin{thebibliography}{9} 

\bibitem{HKKS17} Juraj Hromkovi{\v{c}}, Tobias Kohn, Dennis Komm, Giovanni
Serafini: \textit{Algorithmic Thinking from the Start}, Bulletin of EATCS, Vol.
1, Number 121, 2017. 

\bibitem{IR} Department of Computer Science: \textit{Guidelines Bachelor's Thesis}, available at \url{https://www.inf.ethz.ch/studies/forms-and-documents.html}

\bibitem{SET} Davis, Benjamin Lent, and Maclagan, Diane: \textit{The card game
set}, the Mathematical Intelligencer, 25, No. 3, 2003, 33-40, available at
\url{http://homepages.warwick.ac.uk/staff/D.Maclagan/papers/set.pdf} (last
visit: July 31, 2019).

\bibitem{STBLUM} Blum, Sonja Tabea: \textit{A platform independent,
computer-based learning environment to a textbook for computer science}. Master
Thesis, ETH Zurich, Department of Computer Science, 2018, available at
\url{https://doi.org/10.3929/ethz-b-000312911} (last visit: July 31, 2019).

\bibitem{EFFG} Alsawaier, Raed. (2017). \textit{The Effect of Gamification on
Motivation and Engagement}. International Journal of Information and Learning
Technology. 35. 00-00. 10.1108/IJILT-02-2017-0009. 

\bibitem{GAMTOT} Elshiekh, R. and Butgerit, L. (2017) \textit{Using Gamification
to Teach Students Programming Concepts}. Open Access Library Journal, 4: e3803.
\url{https://doi.org/10.4236/oalib.1103803} (last visit: July 31, 2019)

\bibitem{GAM} Healey, Deborah: \textit{Gamification}, available at
\url{https://www.macmillaneducation.es/wp-content/uploads/2019/04/Gamification-White-Paper_Mar-2019.pdf}
(last visit: July 31, 2019)

\bibitem{GAMREB} Rabah, Jihan \& Cassidy, Robert \& Beauchemin, Robert. (2018).
\textit{Gamification in education: Real benefits or edutainment?}.
10.13140/RG.2.2.28673.56162.

\bibitem{GEFF} Papp, T.A. (2018). \textit{Gamification Effects on Motivation and
Learning: Application to Primary and College Students}.

\bibitem{HTGAM} Morschheuser, Benedikt \& Werder, Karl \& Hamari, Juho \& Abe,
Julian. (2017). \textit{How to Gamify? A Method For Designing Gamification}.
10.24251/HICSS.2017.155. 

\bibitem{CT} Wing, Jennette M. (2006) \textit{Computational Thinking}. Available
at \url{https://www.cs.columbia.edu/~wing/publications/Wing06.pdf} (last visit:
July 31, 2019)

\bibitem{SKAMP} Kamp, Sarah Eleonora: \textit{A platform independent,
computer-based learning environment}. Bachelor Thesis, ETH Zurich, Department of
Computer Science, 2019

\bibitem{TANGK} Tang, Kevin: \textit{Graphs, trees and discrete optimizations:
Computer-based learning environment about combinatorial problems for secondary
education}. Bachelor Thesis, ETH Zurich, Department of Computer Science, 2019

\bibitem{JWEBER} Weber, Jil: \textit{Eine interaktive Lernplattform f\"ur
algorithmisches Denken, f\"ur 4-8 j\"ahrige Kinder}. Master Thesis, ETH Zurich,
Department of Computer Science, 2019

\end{thebibliography}

\vspace{1cm}
\noindent Zurich, \today 

\vspace{2.0cm}
\noindent \student \hfill \cosupervisor \hfill \professor \hfill


\end{document}
